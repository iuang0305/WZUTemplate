%---------------------------------------------------------------------------%
%-                                                                         -%
%-                           LaTeX Template                                -%
%-                                                                         -%
%---------------------------------------------------------------------------%
%- Copyright (C) Huangrui Mo <huangrui.mo@gmail.com> 
%- This is free software: you can redistribute it and/or modify it
%- under the terms of the GNU General Public License as published by
%- the Free Software Foundation, either version 3 of the License, or
%- (at your option) any later version.
%---------------------------------------------------------------------------%
%->> Document class declaration
%---------------------------------------------------------------------------%
\documentclass[doublesided%,fontset=%adobe%fandol
,12pt]{Style/ucasthesis}%
%- Multiple optional arguments:
%- [<singlesided|doublesided|printcopy>]% set one or two sided eprint or print
%- [showmj]% show confidential information
%- [draftversion]% show draft version information
%- [fontset=<fandol|...>]% specify font set to replace automatic detection
%- [scheme=plain]% thesis writing of international students
%- [standard options for ctex book class: draft|paper size|font size|...]%
%---------------------------------------------------------------------------%
%->> Document settings
%---------------------------------------------------------------------------%
\usepackage[super,myhdr,list]{Style/artratex}% document settings
%- usage: \usepackage[option1,option2,...,optionN]{artratex}
%- Multiple optional arguments:
%- [bibtex|biber]% set bibliography processor and package
%- [<numbers|super|authoryear|alpha>]% set citation and reference style
%- <numbers>: textual: Jones [1]; parenthetical: [1]
%- <super>: textual: Jones superscript [1]; parenthetical: superscript [1]
%- <authoryear>: textual: Jones (1995); parenthetical: (Jones, 1995)
%- <alpha>: textual: not available; parenthetical: [Jon95]
%- [geometry]% reconfigure page layout via geometry package
%- [lscape]% provide landscape layout environment
%- [myhdr]% enable header and footer via fancyhdr package
%- [color]% provide color support via xcolor package
%- [background]% enable page background
%- [tikz]% provide complex diagrams via tikz package
%- [table]% provide complex tables via ctable package
%- [list]% provide enhanced list environments for algorithm and coding
%- [math]% enable some extra math packages
\usepackage{Style/artracom}% user defined commands
\usepackage{appendix}
\usepackage{pdfpages}
\usepackage{setspace}
\usepackage{enumerate}
\usepackage{enumitem}
\usepackage{amsthm}
\usepackage{amsmath}
%\usepackage{bicaption} % package for binary captions
%\captionsetup[figure][bi-second]{name=Figure} % the second figure caption name
%\captionsetup[table][bi-second]{name=Table} % the second table caption name
\renewcommand{\theequation}{\arabic{chapter}-\arabic{equation}}
\renewcommand{\thefigure}{\arabic{chapter}-\arabic{figure}}
\renewcommand{\thetable}{\arabic{chapter}-\arabic{table}}
\renewcommand{\contentsname}{目录\vspace{2cm}}
\renewcommand{\thesection}{\arabic{section}}
\renewcommand{\theequation}{ A.\arabic{equation}}%设置附录中公式显示样式
\setcounter{tocdepth}{3}           %增加目录深度
\setcounter{secnumdepth}{3}        %增加编号深度
       
%---------------%此处代码为修改目录层次显示不合适问题--------      
\makeatletter
\renewcommand*\l@section{\@dottedtocline{2}{1.8em}{1.2em}}
\renewcommand*\l@subsection{\@dottedtocline{3}{3.2em}{1.2em}}
\renewcommand*\l@subsubsection{\@dottedtocline{4}{5.2em}{1.2em}}
\makeatother
%--------------------数学环境——————————————————————————————————————————-
\newtheorem{theorem}{\hspace{2em}定理} [section]
\newtheorem{axiom}{\hspace{2em}公理} [section]
\newtheorem{lemma}{\hspace{2em}引理}[section]
\newtheorem{corollary}[]{\hspace{2em}推论}[section]
\newtheorem{assertion}{\hspace{2em}断言}[section]
\newtheorem{proposition}{\hspace{2em}命题} [section]
%\newtheorem{proof}{\hspace{2em}证明} 
\newtheorem{definition}{\hspace{2em}定义}
\newtheorem{example}{\hspace{2em}例子} [section]
\newtheorem{remark}{\hspace{2em}注}[section]
%-
%-----------------------设置默认字体和行距----------------
\usepackage{xeCJK}   %中文字体
\setCJKfamilyfont{kaitibold}{Kaiti SC Bold}
\newcommand{\kaitiB}{\CJKfamily{kaitibold}}   %楷体加粗
\setCJKfamilyfont{heitibold}{Source Han Sans CN}
\newcommand{\heitiB}{\CJKfamily{heitibold}}   %黑体加粗
\setCJKfamilyfont{songtibold}{Songti SC}
\newcommand{\songtiB}{\CJKfamily{songtibold}}   %宋体加粗
 \zihao{-4}\linespread{1.067}\selectfont
 %-------------------设置字体大小--------------
 %   命令  	大小(bp)     	意义
 %\zihao{0} 	42      	初号
 %\zihao{-0}	36          小初
 % \zihao{1}	26        	一号
 % \zihao{-1}	24      	小一
 % \zihao{2}	22      	二号
 % \zihao{-2}	18      	小二
 % \zihao{3}	16	        三号
 % \zihao{-3}	15	        小三
 % \zihao{4}	14      	四号
 % \zihao{-4}	12      	小四
 % \zihao{5}	10.5      	五号
 % \zihao{-5}	9         	小五
 % \zihao{6}	7.5	        六号
 % \zihao{-6}	6.5	        小六
 % \zihao{7}	5.5     	七号
 % \zihao{8}	5   	    八号
 


%---------------------------------------------------------------------------%
%------------定义页面大小----------------
%\usepackage{geometry}\geometry{	a4paper,	total={170mm,257mm},	left=20mm,	top=20mm,}
%->> Document inclusion
%---------------------------------------------------------------------------%
%\includeonly{Tex/Chap_1,...,Tex/Chap_N}% selected files compilation
%---------------------------------------------------------------------------%
%->> Document content
%---------------------------------------------------------------------------%
\begin{document}
%-
%-> Frontmatter: title page, abstract, content list, symbol list, preface
%-
\frontmatter% initialize the environment
%---------------------------------------------------------------------------%
%->> Titlepage information
%---------------------------------------------------------------------------%
%-
%\includepdfmerge{[pdfname],[a-b]}//pdfname 是pdf的名字,放当前目录下,a-b是引用的页数。
%
%\includepdfmerge{lwfmmm.pdf,1}
\includepdfmerge{fengmian.pdf,1-2}
%-> Chinese titlepage
%-
\confidential{}% confidential level
\schoollogo{scale=0.95}{wzu}% university logo
%\title{在光学腔中自旋压缩态产生的研究}% 
\title[温州大学硕士学位论文\LaTeX{}模板]{温州大学学位论文\LaTeX{}模板}
\author{刘刚}% name of author
\advisor{王明锋}% supervisor
\advisorsec{}% co-supervisor
\degree{硕士}% degree
\degreetype{理学}% degree type
\major{凝聚态物理}% major
\institute{温州大学}% institute of author
\chinesedate{2014~年~6~月}% customized date, 6 for summer and 12 for winter graduation
%-
%-> English titlepage
%-
\englishtitle{\LaTeX{} Thesis Template\\ of \\ WenZhou University}
\englishauthor{Huangrui Mo}
\englishadvisor{Supervisor: Professor Qingquan Liu}
\englishdegree{Master of Natural Science}% degree type <Doctor|Master> of <Philosophy|Natural Science|Engineering>
\englishthesistype{thesis}% thesis type <thesis|dissertation>
\englishmajor{Fluid Mechanics}% major
\englishinstitute{Institute of Mechanics, Chinese Academy of Sciences}
\englishdate{June, 2014}% customized date
%-
%-> Create titlepages
%-
%\maketitle
%\makeenglishtitle
%-
%-> Author's declaration
%-
\makedeclaration
%-
%-> Chinese abstract
%-

\chapter[摘要]{温州大学学位论文\LaTeX{}模板}
\vbox{}

\section*{\qquad\qquad\qquad\qquad\qquad\qquad\quad\  摘\quad 要}
\vbox{}
%\chaptermark{摘\quad 要}

\setcounter{page}{1}% set page number
\pagenumbering{Roman}% set large roman

{
	\par
	\zihao{4}
	\linespread{1.5}\selectfont
	
	
	本文是温州大学硕士学位论文模板,根据中国科学院大学学位论文模板ucasthesis修改而来,并依据其实用说明写了这份使用说明文档。主要内容为介绍\LaTeX{}文档类ucasthesis的用法,以及如何使用\LaTeX{}快速高效地撰写学位论文。
	
	
\par

\vbox{}
\keywords{温州大学,学位论文,\LaTeX{}模板}% 中文关键词
%-
%-> English abstract
%-

\chapter[Abstract]{\linespread{1.5}\selectfont \LaTeX{} Thesis Template of WenZhou University}

\vbox{}
\section*{\qquad\qquad\qquad\qquad\qquad\quad\quad ABSTRACT}
\vbox{}
%\chaptermark{Abstract}
%\vskip 1cm


{
	\par
	\zihao{4}
	\linespread{1.5}\selectfont
	
This paper is a help documentation for the \LaTeX{} class ucasthesis, which is  a thesis template for the University of WenZhou, which come from a thesis template for the University of Chinese Academy of Sciences. The main content is about how to use the ucasthesis, as well as how to write thesis efficiently by using \LaTeX{}.
	
	\par

\vbox{}
%\vskip 1cm
\englishkeywords{\linespread{1.5}\selectfont WenZhou University, Thesis, \LaTeX{} Template}% 英文关键词
%---------------------------------------------------------------------------%
% 摘要在这里修改
{% content list region
\linespread{1.2}% local line space
%
\tableofcontents% 显示目录
%\intotoc{\listfigurename}% 这里决定论文中是否显示图片索引
%\listoffigures% figures catalog
%\intotoc{\listtablename}% 这里决定论文中是否显示图表索引
%\listoftables% tables catalog
}
%\input{Tex/Prematter}% 这里决定论文中是否显示符号列表,对应文件可添加符号
%-
%-> Mainmatter
%-
\mainmatter% initialize the environment
%---------------------------------------------------------------------------%
%->> Main content
%---------------------------------------------------------------------------%
\input{Tex/Chap_Intro}
\input{Tex/Chap_Guide}
\chapter{撰写要求及说明}
\vbox{}\vbox{}
\section{撰写要求}
学位论文是研究生科研工作成果的集中体现,是评判学位申请者学术水平、授予其学位的主要依据,是科研领域重要的文献资料。根据《科学技术报告、学位论文和学术论文的编写格式》(GB/T 7713-1987)、《学位论文编写规则》(GB/T 7713.1-2006)和《文后参考文献著录规则》(GB7714—87)等国家有关标准,特制订本规定。

\section{论文封面设置}
此模板暂时未能实现封面格式的封装,因此需要单独制作封面并导入到\LaTeX{}中。具体制作方法为,在Word中制作好封面以后打印完PDF文件,再将PDF文件导入到\LaTeX{}即可。在源文件中已有fengmian.doc文件,在使用过程中只需要打开此文档填入自己的论文信息即可,随后使用虚拟打印机打印为fengmian.pdf文件(导入代码已设置完毕,无需关心导入)保存即可(建议使用adobe printer打印)。

正文中其它的文件格式已按照温大硕士毕业论文Word模板设置完毕,因此在使用该软件的过程中无需关心格式,只需关心文档的写作即可,因为\LaTeX{}的主要优势除了生成高质量的PDF文件以外,另一大优势就是格式文档分离。


\section{测试生僻字}

霜蟾盥薇曜灵霜颸妙鬘虚霩淩澌菀枯菡萏泬寥窅冥毰毸濩落霅霅便嬛岧峣瀺灂姽婳愔嫕飒纚棽俪緸冤莩甲摛藻卮言倥侗椒觞期颐夜阑彬蔚倥偬澄廓簪缨陟遐迤逦缥缃鹣鲽憯懔闺闼璀错媕婀噌吰澒洞阛闠覼缕玓瓑逡巡諓諓琭琭瀌瀌踽踽叆叇氤氲瓠犀流眄蹀躞赟嬛茕頔璎珞螓首蘅皋惏悷缱绻昶皴皱颟顸愀然菡萏卑。

暒墌墍墎墏墐墒墒墓墔墕墖墘墖墚墛坠墝增墠墡墢墣墤墥墦墧墨墩墪樽墬墭堕墯墰墱墲坟墴墵垯墷墸墹墺墙墼墽垦墿壀壁壂壃壄壅壆坛壈壉壊垱壌壍埙壏壐壑壒压壔壕壖壗垒圹垆壛壜壝垄壠壡坜壣壤壥壦壧壨坝塆圭嫶嫷嫸嫹嫺娴嫼嫽嫾婳妫嬁嬂嬃嬄嬅嬆嬇娆嬉嬊娇嬍嬎嬏嬐嬑嬒嬓嬔嬕嬖嬗嬘嫱嬚嬛嬜嬞嬟嬠嫒嬢嬣嬥嬦嬧嬨嬩嫔嬫嬬奶嬬嬮嬯婴嬱嬲嬳嬴嬵嬶嬷婶嬹嬺嬻嬼嬽嬾嬿孀孁孂娘孄孅孆孇孆孈孉孊娈孋孊孍孎孏嫫婿媚嵭嵮嵯嵰嵱嵲嵳嵴嵵嵶嵷嵸嵹嵺嵻嵼嵽嵾嵿嶀嵝嶂嶃崭嶅嶆岖嶈嶉嶊嶋嶌嶍嶎嶏嶐嶑嶒嶓嵚嶕嶖嶘嶙嶚嶛嶜嶝。

嶞嶟峤嶡峣嶣嶤嶥嶦峄峃嶩嶪嶫嶬嶭崄嶯嶰嶱嶲嶳岙嶵嶶嶷嵘嶹岭嶻屿岳帋巀巁巂巃巄巅巆巇巈巉巊岿巌巍巎巏巐巑峦巓巅巕岩巗巘巙巚帠帡帢帣帤帨帩帪帬帯帰帱帲帴帵帷帹帺帻帼帽帾帿幁幂帏幄幅幆幇幈幉幊幋幌幍幎幏幐幑幒幓幖幙幚幛幜幝幞帜幠幡幢幤幥幦幧幨幩幪幭幮幯幰幱庍庎庑庖庘庛庝庠庡庢庣庤庥庨庩庪庬庮庯庰庱庲庳庴庵庹庺庻庼庽庿廀厕廃厩廅廆廇廋廌廍庼廏廐廑廒廔廕廖廗廘廙廛廜廞庑廤廥廦廧廨廭廮廯廰痈廲廵廸廹廻廼廽廿弁弅弆弇弉弖弙弚弜弝弞弡弢弣弤弨弩弪弫弬弭弮弰弲弪弴弶弸弻弼弽弿彖彗彘彚。

%\input{Tex/Chap_5}
%\input{Tex/Chap_6}
%---------------------------------------------------------------------------%
% main content
%-
%-> Appendix
%-
%\cleardoublepage%
%\appendix% initialize the environment
%\input{Tex/Appendix}% appendix content
%-
%-> Backmatter: bibliography, glossary, index
%-
\backmatter% initialize the environment
%-
%\appendix% initialize the environment  ------如果要求附录在参考文献前面,则打开这个开关
%\input{Tex/Appendix}% appendix content
%-
\intotoc{\bibname}% add link to contents table and bookmark
%\input{Tex/reference}% other information
\bibliography{Biblio/ref}% bibliography
%---------------------------------------------------------------------------%
%->> Backmatter
%---------------------------------------------------------------------------%
\chapter{作者简历及攻读学位期间发表的学术论文与研究成果}

%\textbf{本科生无需此部分}。

\section*{作者简历}

\subsection*{wzuthesis作者}

刘刚,甘肃省礼县人,温州大学数理学院凝聚态物理2019届硕士研究生。

(希望后面有人可以维护此项目,通过修改或重写使的该模板更加完善,为后面的师弟师妹们扫平障碍!)
\subsection*{casthesis作者}
%
吴凌云,福建省屏南县人,中国科学院数学与系统科学研究院博士研究生。

\subsection*{ucasthesis作者}

莫晃锐,湖南省湘潭县人,中国科学院力学研究所硕士研究生。

\section*{已发表(或正式接受)的学术论文:}

{
%\setlist[enumerate]{}% restore default behavior
\begin{itemize}[nosep]
    \item ucasthesis: A LaTeX Thesis Template for the University of Chinese Academy of Sciences, 2014.
\end{itemize}
}

\section*{申请或已获得的专利:}

(无专利时此项不必列出)

\section*{参加的研究项目及获奖情况:}

可以随意添加新的条目或是结构。

\chapter[致谢]{致\quad 谢}\chaptermark{致\quad 谢}% syntax: \chapter[目录]{标题}\chaptermark{页眉}
%\thispagestyle{noheaderstyle}% 如果需要移除当前页的页眉
%\pagestyle{noheaderstyle}% 如果需要移除整章的页眉

感激casthesis作者吴凌云前辈,gbt7714-bibtex-style
开发者zepinglee,和ctex众多开发者们。若没有他们的辛勤付出和非凡工作,\LaTeX{}菜鸟的我是无法完成此温州大学硕士学位论文\LaTeX{}模板(wzuthesis,修改好以后用这个文件名)的。在\LaTeX{}中的一点一滴的成长源于开源社区的众多优秀资料和教程,在此对所有\LaTeX{}社区的贡献者表示感谢!

在当时写论文修改模板过程中遇到了各种各样的问题,此\LaTeX{}模板的最终成型离不开\textbf{\LaTeX{}技术交流1群}(群号:91940767,付费进群)各位群友以及 \href{https://wenda.latexstudio.net/}{\textbf{\LaTeX{}问答社区}}的帮助,在此对他们的认真无私的付出表示感谢。

\cleardoublepage[plain]% 让文档总是结束于偶数页,可根据需要设定页眉页脚样式,如 [noheaderstyle]
%---------------------------------------------------------------------------%
% other information
\end{document}
%---------------------------------------------------------------------------%

